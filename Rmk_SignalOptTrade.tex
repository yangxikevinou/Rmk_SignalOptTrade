\documentclass[openany,oneside]{article}
\usepackage{fullpage}
\usepackage{amsmath,amsthm,amssymb,mathtools,tikz,epsfig,enumerate}
\usepackage{hyperref,titling,titlesec,pdfpages,setspace,fancyhdr,multicol,appendix}
\usepackage[numbers]{natbib}

% formatting
\setlength{\parskip}{2ex}
\setlength{\parindent}{0pt}

% statement environment
\newtheorem{thm}{Theorem}[section]
\newtheorem{prop}[thm]{Proposition}
\newtheorem{lem}[thm]{Lemma}
\newtheorem{cor}[thm]{Corollary}
 
\theoremstyle{definition}
\newtheorem{defn}[thm]{Definition}
\newtheorem{prob}[thm]{Problem}
\newtheorem{eg}[thm]{Example}
 
\theoremstyle{remark}
\newtheorem{rem}[thm]{Remark}

% shorthand notations
\newcommand{\E}{\mathbb{E}} % expectation
\renewcommand{\P}{\mathbb{P}} % probability
\newcommand{\I}{\mathbb{I}} % indicator
\newcommand{\F}{\mathcal{F}} % filtration
\DeclarePairedDelimiter{\abs}{\lvert}{\rvert} % absolute value, or cardinality
\DeclarePairedDelimiter{\norm}{\lVert}{\rVert} % norm
\newcommand{\ts}{\textstyle}
\DeclareMathOperator{\esssup}{ess\, sup}
\newcommand*{\remaend}{\hfill\text{$\diamond$}}
\newcommand{\close}{\hspace*{\fill}$\diamond$}
\newcommand{\closeEqn}{\tag*{$\diamond$}}
\newcommand{\de}{\,\mathrm{d}}

% info
\title{Rmk_SignalOptTrade}
\date{Spring 2018}
\hypersetup{bookmarksnumbered=true, 
			bookmarksopen=true,
            unicode=true,
            pdftitle=\thetitle,
            pdfauthor=Yangxi Ou,
            pdfstartview=FitH,
            pdfpagemode=UseOutlines}
%%%%%%%%%%%%%%%%%%%%%%%%%%%%%



\begin{document}

% title
\begin{center}	
	\textbf{\Large Remarks on Incorporating Signals into Optimal Trading by Lehalle and Neuman} \\*[5ex]
    \thedate \\*[5ex]
\end{center}


% introduction
\section{Introduction}
Artificial price limits are sometimes enforced in the financial market due to regulatory powers, for example the exchange rate of Swiss Franc in Euro (EUR/CHF) is capped at a known threshold by the Swiss central bank. Such limits should be taken into account when acquiring or liquidating positions, in addition to price impact, transaction costs, and inventory risk. We study the optimal liquidation problem of an asset with one-sided price limit, following \cite{lehalle2017incorporating}.


% model setup
\section{Model setup}
Fix a filtered probability space $(\Omega, \F, \{\F_t\}_{t\in[0,T]}, \P)$ satisfying the usual conditions. Denote the conditional expectation $\E[\cdot \vert \F_t]$ by $\E_t$ for $t\in[0,T]$. Also define the following two spaces of stochastic processes:
\begin{align*}
\mathcal{H}^1 :=\Bigl\{ & S \textrm{ adapted c\'adl\'ag} : S=M+A \textrm{ is the canonical decomposition, } \\
& \hspace{1em}\textrm{where $M$ is the (local) martingale component and $A$ is the predictable component, and } \\
& \ts\hspace{2em}\E\bigl[\langle M\rangle^{1/2}_T\bigr] + \E\bigl[\int_0^T \abs{\de A_t}\bigr] < \infty\Bigr\}, \textrm{ and} \\
\mathcal{V} := \Bigl\{ & v \textrm{ progressively measurable} : \int_0^T \abs{v_t} \de t < \infty \quad \P\textrm{-a.s.} \Bigr\}
\end{align*}
%$\mathcal{H}^1:=\{S \textrm{ adapted c\'adl\'ag} : \E S^\ast_T = \E[\sup_{t\in[0,T]} \abs{S_t}] < \infty\}$.

Let $P$, the asset price, be a semi-martingale in $\mathcal{H}^1$. Denote the position in the asset at time $t$ by $X_t$, where the initial position is given as $X_0:=x$. The investor controls the liquidation intensity $-\dot{X} =: u \in \mathcal{V}$. Given the transaction cost coefficient $\lambda>0$, the penalty coefficients for running and terminal inventory $\gamma>0$ and $\Gamma>0$, respectively, the investor aims to maximize the inventory risk adjusted profit and loss of liquidation:
\[
\ts V^u:= \E\left[\underbrace{\int_0^T \overbrace{(P_t - \lambda u_t)}^{\textrm{execution price}} u_t \de t}_{\textrm{terminal cash position}} + \underbrace{P_T X_T}_{\stackrel{\textrm{terminal}}{\textrm{asset position}}} - \underbrace{\int_0^T \gamma X_t^2 \de t}_{\stackrel{\textrm{running}}{\textrm{inventory penalty}}} - \underbrace{\Gamma X_T^2}_{\stackrel{\textrm{terminal}}{\textrm{inventory penalty}}} \right].
\]
Sometimes it is convenient to consider the dynamic version of the problem, so we define the controlled value process $t\mapsto V^u_t$ and the maximum value process $t\mapsto V^\ast_t$ as $\forall t\in[0,T]$,
\begin{align*}
&\ts V^u_t := \E_t \left[\int_t^T (P_s-\lambda u_s)u_s \de s + P_T X_T -\int_t^T \gamma X_s^2 \de s - \Gamma X_T^2 \right], \textrm{ and} \\
&\ts V^\ast_t := \esssup_{(u_s)_{s\in[t,T]}\in\mathcal{V}} V^u_t.
\end{align*}
This setup is more general than that in \cite{lehalle2017incorporating}, which requires $\de P_t = I_t \de t + \de M_t$ where $I$ is a time-homogeneous continuous Markov process.

\begin{eg}
The main application we have in mind for this general setup is the case of a price cap. Let $c\in\mathbb{R}$ be the price cap (imposed by the regulator) of the asset price $P$, i.e.\ $P:=S-(M-c)_+$, where $S\in\mathcal{H}^1$ is a martingale and $M_t:=\sup_{s\in[0,t]} S_s$ is its running maximum. Note that $P_t\le c$ with equality iff $S_t=M_t\ge c$. Such formulation of the price cap reflects the reality of minimal regulatory intervention in the market.\close
%However, we will not impose this assumption in the next section in which we solve the general optimization problem.
\end{eg}


% solution
\section{Solution to the general problem}
In our general - not necessarily Markovian - setup, the dynamic programming and differential equation technique in \cite{lehalle2017incorporating} no longer applies, which is why we follow a calculus of variation approach similar to \cite{bank2017hedging} and \cite{bouchard2017equilibrium}. It is immediate that the functional $u\mapsto V^u$ is quadratic (including linear and constant terms) with a strictly negative definite quadratic coefficient, and hence there exists a unique maximum characterized by the critical point $u^\ast$ with $\frac{\delta V^u}{\delta u}(u^\ast) =0$, where $\frac{\delta V^u}{\delta u}$ is the G\^ateaux derivative. A rigorous treatment can be found in \cite{ekeland1999convex}. This approach leads to a system of forward backward stochastic differential equations (FBSDE) characterizing the optimal trading rate and portfolio holding, where the forward equation is given by the dynamics $-\dot{X}=u$ and the backward equation is given by the optimality condition $\frac{\delta V^u}{\delta u}=0$. Our FBSDE differs from those of \cite{bank2017hedging} and \cite{bouchard2017equilibrium} in terminal conditions and non-martingale asset prices. Fortunately, the system is linear and can be solved explicitly. Once the optimal selling rate $u^\ast$ is known, the maximum value $V^\ast$ is then directly computed by definition. The main result is stated as follows.

\begin{prop}[Formulae for optimal strategy and value function]\label{main}
Set $\beta:=\sqrt{\lambda^{-1}\gamma}$ as well as
\begin{align*}
&\ts G(t)  := \beta\cosh(\beta t)+\lambda^{-1}\Gamma\sinh(\beta t),\\
&\ts v_2(t):= -(\log G)'(T-t) = -\frac{G'(T-t)}{G(T-t)},\\
&\ts v_1(t):= \frac{1}{2\lambda}\int_t^T \frac{G(T-s)}{G(T-t)} \de_s(\E_t P_s) = \frac{1}{2\lambda}\int_t^T \exp\left(\int_t^s v_2\right) \de_s(\E_t P_s), \\
&\ts v_0(t):= \int_t^T \E_t[v_1^2(s)] \de s.
\end{align*}
The unique maximizer $u^\ast$ of $u\mapsto V^u$ solves the linear ordinary differential equation (ODE)
\begin{align*}
&\ts u^\ast_t = -v_2(t) X_t^{\ast} - v_1(t), \textrm{ where } X_t^{\ast} := X_t^{u^\ast} \textrm{ and } u^\ast_t = -\dot{X}^\ast_t,
\end{align*}
so that the optimal liquidation trajectory is given by
\begin{align*}
&\ts X^{\ast}_t = \frac{G(T-t)}{G(T)}x + \int_0^t \frac{G(T-t)}{G(T-s)} v_1(s) \de s.
\end{align*}
Finally, the maximum value process $V^\ast_t$ is given by
\begin{align*}
&\ts V^\ast_t = P_t X_t^{\ast} + \lambda\left[v_0(t) + 2 v_1(t) X_t^{\ast} + v_2(t) (X_t^{\ast})^2\right]. \closeEqn
\end{align*}
\end{prop}

\begin{rem}\label{RemIntegral}
Note that $s\mapsto \E_t P_s$ is of finite variation and hence the integral in the definition of $v_1(t)$ makes sense. Indeed, since $P\in \mathcal{H}^1$, it admits a unique Doob-Meyer decomposition $P=A+M$ in which $A\in\mathcal{H}^1$ is predictable and of finite variation and $M\in\mathcal{H}^1$ is a martingale. Thus by definition,
\[
 \ts \sum_i \abs{\E_t P_{s_i} - \E_t P_{s_{i-1}}} \le \E_t\left[\sum_i \abs{A_{s_i} - A_{s_{i-1}}}\right] \le \E_t \int_t^T \abs{\de A_s} < \infty.\closeEqn
\]
\end{rem}

\begin{proof}[Proof of Proposition~\ref{main}]
We follow \cite{bouchard2017equilibrium} for the proof. For convenience, we only maximize $V^u_0$, i.e.\ we restrict our attention to the case $t=0$, and the general case follows analogously. Recall that $X_t = x-\int_0^t u_s \de s$, so $X$ is an affine function of $u$. It is straightforward that $V^u_0$ is a quadratic functional (with linear and constant terms) of $(u,X)$ with strictly negative definite quadratic coefficients, and hence it admits a unique maximizer characterized by the critical point. We will solve for the critical point in feedback form directly.

Step 1: Compute the G\^ateaux derivative. We fix a direction of variation $\alpha \in \mathcal{V}$ and compute
\begin{align*}
\ts\left\langle \frac{\delta V^u_0}{\delta u}, \alpha \right\rangle &\ts= \frac{d}{d\epsilon}\Big\vert_{\epsilon=0} V^{u+\epsilon\alpha}_0  \\
&\ts= \E\left[\int_0^T P_t \alpha_t \de t - 2\lambda\int_0^T u_t \alpha_t \de t - P_T \int_0^T \alpha_t \de t + 2\gamma\int_0^T X_t \int_0^t \alpha_s \de s \de t + 2\Gamma X_T \int_0^T \alpha_t \de t \right] \\
&\ts= \E\int_0^T \alpha_t \E_t\left[P_t - 2\lambda u_t - P_T + 2\gamma\int_t^T X_s \de s + 2\Gamma X_T \right] \de t.
\end{align*}
Setting $\frac{\delta V^u_0}{\delta u}=0$, we obtain $\E_t[-2\lambda u_t + P_t - P_T + 2\gamma\int_t^T X_s \de s + 2\Gamma X_T ] = 0$. Combining this with the dynamics of $X$, we obtain the following linear system of forward backward stochastic differential equations (FBSDE):
\[
\begin{cases}
X_t &= x-\int_0^t u_s \de s \\
u_t &= \frac{1}{2\lambda}\E_t\left[P_t-P_T+2\gamma\int_t^T X_s \de s + 2\Gamma X_T\right] = \frac{1}{2\lambda}\left(P_t-2\gamma\int_0^t X_s \de s - N_t\right), \textrm{ where $N$ is a martingale.} 
\end{cases}
\]
Rewriting the system in vector form, we have
\[
\begin{pmatrix} X_t \\ u_t \end{pmatrix} = \begin{pmatrix} x \\ u_0 \end{pmatrix} + \int_0^t \begin{pmatrix} 0 & -1 \\ -\lambda^{-1}\gamma & 0 \end{pmatrix} \begin{pmatrix} X_s \\ u_s \end{pmatrix} \de s + \frac{1}{2\lambda}\begin{pmatrix} 0 \\ P_t - N_t \end{pmatrix}
\]
with $u_T = \lambda^{-1}\Gamma X_T$.

Step 2: Solve the FBSDE for the critical point. For notational ease, let
\[
 y:=\begin{pmatrix} X \\ u\end{pmatrix},\qquad z:=\begin{pmatrix} 0 \\ P-N\end{pmatrix},\qquad\text{and}\qquad B:=\begin{pmatrix} 0 & -1 \\ -\lambda^{-1}\gamma & 0\end{pmatrix},
\]
so that the equation reads $y_t = y_0 + \int_0^t B y_s ds + (2\lambda)^{-1}z_t$ and $(\lambda^{-1}\Gamma, -1)y_T = 0$. This is a linear stochastic differential equation in $y$ and well-known to have the solution
\[
\ts y_T = \exp((T-t)B)y_t + \frac{1}{2\lambda} \int_t^T \exp((T-s)B) dz_s.
\]
Thus,
\[
\ts 0 = q(T-t)y_t + \frac{1}{2\lambda}\int_t^T q(T-s) dz_s,\qquad \textrm{ where } q(t):=(\lambda^{-1}\Gamma, -1)\exp(tB).
\]
Note that $q$ satisfies the vector-valued differential equation $\dot{q}=qB$ with $q(0)=(\lambda^{-1}\Gamma, -1)$. Rewriting it in scalar-valued form, we obtain a linear second-order differential equation as follows:
\begin{align*}
\begin{cases} (\dot{q_1},\dot{q_2})=(-\lambda^{-1}\gamma q_2, -q_1) \\ q_1(0)=\lambda^{-1}\Gamma, q_2(0)=-1 \end{cases}
\Rightarrow\quad \begin{cases}
\ddot{q_2}=\lambda^{-1}\gamma q_2 \\ q_2(0)=-1, \dot{q_2}(0)=-\lambda^{-1}\Gamma.
\end{cases}
\end{align*}
Set $\beta:=\sqrt{\lambda^{-1}\gamma}>0$. It is well-known and easily checked that the solution is given% by a linear combination of the fundamental solutions hyperbolic trigonometric functions:
\begin{align*}
q_2(t)=q_2(0)\cosh(\beta t)+\dot{q_2}(0)\beta^{-1}\sinh(\beta t) = -\cosh(\beta t)-\lambda^{-1}\Gamma\beta^{-1}\sinh(\beta t).
\end{align*}
Then from $q_1=-\dot{q_2}$, we also get $q_1(t)=\beta\sinh(\beta t)+\lambda^{-1}\Gamma\cosh(\beta t)$. Hence, we can express $u^\ast$ via
\begin{align*}
&\ts 0=q_1(T-t)X^\ast_t+q_2(T-t)r^\ast_t+\frac{1}{2\lambda}\int_t^T q_2(T-s) \de (P_s-N_s) \\
\Rightarrow\quad &\ts u^\ast_t = -\frac{q_1(T-t)}{q_2(T-t)} X^\ast_t - \frac{1}{2\lambda} \E_t\int_t^T\frac{q_2(T-s)}{q_2(T-t)} \de P_s \\
\Rightarrow\quad &\ts u^\ast_t = \frac{G'(T-t)}{G(T-t)} X^\ast_t - \frac{1}{2\lambda} \int_t^T \frac{G(T-s)}{G(T-t)} \de _s(\E_t P_s), \textrm{ with } G(t):=-\beta q_2(t) = \beta\cosh(\beta t)+\lambda^{-1}\Gamma\sinh(\beta t).
\end{align*}
In the last equality we use integration by parts to move expectation to the integrator; the integral can be understood as a pathwise Lebesgue-Stieltjes integral thanks to Remark~\ref{RemIntegral} above. Since both $G$ and $G'$ are bounded from above and below away from zero, we have $\E\abs{u^\ast_t} \le C_1 (x+\int_0^t \E\abs{u^\ast_s} \de s) + C_2$ for some $C_1, C_2 > 0$. Applying Gronwall's lemma, we find that $t\mapsto u^\ast_t$ is bounded in $L^1(\Omega)$ and hence admissible.

Defining $v_2$ and $v_1$ as in the statement of the proposition, we have $u^\ast_t = -v_2(t)X^\ast_t -v_1(t)$, which is a linear differential equation (since $u^\ast=-\dot{X^\ast}$), whose solution is given by
\[
\ts X^\ast_t = h(0,t)x + \int_0^t h(s,t)v_1(s) \de s,\qquad \textrm{ where } h(s,t):=\exp\int_s^t v_2 = \exp\int_s^t -\frac{G'(T-\tau)}{G(T-\tau)}\de \tau=\frac{G(T-t)}{G(T-s)}.
\]

Step 3:
% Compute the value function. Given $u^*$ and $X^*$, it is a straightforward exercise to compute
% \[
%  \ts V^\ast_t = P_t X_t^\ast + \lambda\left[v_0(t) + 2 v_1(t) X_t^\ast + v_2(t) (X_t^\ast)^2\right].\qedhere
% \]
Instead of directly plugging the optimal strategy $r^\ast$ into the function $r\mapsto V^r_0$, we leverage on the optimality condition, in particular the following two identities:
\begin{align*}
0 &= \left\langle \frac{\delta V^u_0}{\delta u}(u^\ast), u^\ast \right\rangle = \E\left[\int_0^T P_t u^\ast_t dt - 2\lambda\int_0^T (u^\ast_t)^2 dt - P_T \int_0^T u^\ast_t dt + 2\gamma\int_0^T X^\ast_t \int_0^t u^\ast_s ds dt + 2\Gamma X^\ast_T \int_0^T u^\ast_t dt \right] \\
&= \E\left[\int_0^T P_t u^\ast_t dt -2\lambda \int_0^T (u^\ast_t)^2 dt + P_T(X^\ast_T-x) - 2\gamma\int_0^T X^\ast_t (X^\ast_t-x) dt - 2\Gamma X^\ast_T (X^\ast_T-x) \right] \\
&= \E\left[\int_0^T P_t u^\ast_t dt -2\lambda \int_0^T (u^\ast_t)^2 dt + P_T X^\ast_T - 2\gamma\int_0^T (X^\ast_t)^2 dt - 2\Gamma (X^\ast_T)^2 \right] -x\E\left[P_T - 2\gamma \int_0^T X^\ast_t dt - 2\Gamma X^\ast_T \right], \\
0 &= \E\left[-2\lambda u^\ast_0 + P_0-P_T + 2\gamma\int_0^T X^\ast_t dt + 2\Gamma X^\ast_T\right] = -2\lambda u^\ast_0 + P_0 - \E\left[P_T - 2\gamma \int_0^T X^\ast_t dt - 2\Gamma X^\ast_T \right].
\end{align*}
It follows that
\begin{align*}
V^\ast_0 &= V^{u^\ast}_0 = \E\left[\int_0^T (P_t-\lambda u^\ast_t) u^\ast_t dt + P_T X^\ast_T - \int_0^T \gamma (X^\ast_t)^2 dt - \Gamma (X^\ast_T)^2 \right] \\
&= \frac{1}{2}\E\left[\int_0^T P_t u^\ast_t dt + P_T X^\ast_T \right] + \frac{1}{2}x(-2\lambda u^\ast_0 + P_0) = \frac{1}{2}\E\left[xP_0 + \int_0^T X^\ast_t dP_t \right] + \frac{1}{2}x(2\lambda v_2(0)x + 2\lambda v_1(0) + P_0) \\
&= xP_0 + \lambda(v_2(0)x^2+v_1(0)x) + \frac{1}{2}\E\left[\int_0^T \left(\frac{G(T-t)}{G(T)}x + \int_0^t \frac{G(T-t)}{G(T-s)}v_1(s) ds\right) dP_t\right] \\
&= xP_0 + \lambda(v_2(0)x^2+v_1(0)x) + \lambda x \frac{1}{2\lambda}\E\left[\int_0^T \frac{G(T-t)}{G(T)} dP_t\right] + \lambda \E\left[\int_0^T \frac{1}{2\lambda} \E_s\left[\int_s^T \frac{G(T-t)}{G(T-s)} dP_t \right] v_1(s) ds\right] \\
&= xP_0 + \lambda(v_2(0)x^2+v_1(0)x) + \lambda x v_1(0) + \lambda \E\left[\int_0^T v_1^2(s) ds\right] = xP_0 + \lambda(v_2(0)x^2 + 2v_1(0)x + v_0(0)),
\end{align*}
where $v_0$ is defined as in the statement of the proposition.
\end{proof}

Note that the value function consists of two parts: $xP_0$ is the paper value of the position marking to market, and $\lambda(v_0(0)+2v_1(0)x+v_2(0)x^2)$ is the estimated risk-adjusted implementation shortfall of the meta order under the optimal strategy.

\begin{rem}
Observe that our representation of $V^\ast_t$ and the definitions of $v_i(t)$, $i=0,1,2$, are different from those in \cite{lehalle2017incorporating}. These modifications reflect our FBSDE approach. In particular, $G$ plays the role of ``the fundamental solution'' of a linear differential equation, and the expression of $u^\ast$ in terms of $G$ is a consequence of Duhamel's principle. Of course, our representations of $u^\ast$ and $V^\ast$ do coincide with those of~\cite{lehalle2017incorporating} by uniqueness.

Moreover, we note that, in the setting of \cite{lehalle2017incorporating}, $\de_s(\E_t P_s) = (E_t I_s)\de s$ by Fubini's theorem. In the case of a general semi-martingale asset price, our result simply confirms that the investor only needs to do the same thing as before to achieve the optimum: forecasting price trends until the end of the trading period.\close
\end{rem}



% with price limit
\section{Optimal strategy in the case of a price cap}
Let us now consider the special case in which the price process is a martingale reflected at some level $c$. For this, we shall compute $\E_t P_s$ if $P=S-(M-c)_+$ for some $c\in\mathbb{R}$, $S\in\mathcal{H}^1$ a martingale, and $M_t:=\sup_{s\in[0,t]}S_s$:
\begin{align*}
& \E_t P_s = \E_t[S_s-(M_s-c)_+] = S_t-\E_t(M_s-c)_+ = S_t-(M_t-c)_+ - \E_t\left[(M_s-c)_+ - (M_t-c)_+\right] \\
=& P_t -\left\{ \E_t\left[(M_s-c)_+; M_t\le c\right] + \E_t\left[(M_s-c)-(M_t-c); M_t > c\right] \right\} \\
=& P_t -\left\{ \E_t\left[(M_s-c)_+; M_t\le c\right] + \E_t\left[(M_s-M_t)_+; M_t > c\right] \right\} = P_t - \E_t\left[(M_s-c\vee M_t)_+\right].
\end{align*}
Note that $\de_s(\E_t P_s) = -\de_s\left[\E_t(M_s-c\vee M_t)_+\right] = -\de_s L_s(t)$, where $L_s(t):=\E_t(M_s-c\vee M_t)_+$ is the price at time $t$ of a lookback call option on $S$ maturing at time $s$ with the fixed strike $c\vee M_t$. We thus have reduced the problem of optimal trading with a price cap to the computation of the Theta of a lookback call option.

%\begin{rem}
%Indeed, if $S$ is a time-homogeneous Markov process, then all we need is just the Greek theta of the lookback call. (If $S$ is not time-homogeneous, there is a distinction between the forward and backward time variables and the definition of the Greek theta may be ambiguous.)
%\end{rem}

We devote the following to the computation of $L_s(t)$. One way is to remove the dependency on the running maximum $M$ as follows:
\begin{multline*}
\ts L_s(t) = \E_t(M_s-c\vee M_t)_+ = \E_t \int_{c\vee M_t}^{\infty}\I_{(-\infty,M_s)}(y) \de y = \int_{c\vee M_t}^\infty \E_t \I_{(-\infty,M_s)}(y)\de y \\
 \ts = \int_{c\vee M_t}^\infty \P_t(M_s > y) \de y= \int_{c\vee M_t}^\infty \P_t(\tau_t(y) \le s) \de y,
\end{multline*}
where $\tau_t(y):=\inf\{s\in[t,\infty] : S_s> y\}$ is the first hitting time of the set $(y,\infty)$ after time $t$ by $S$. Note that the identity $\P_t(M_s > y) = \P_t(\tau_t(y)\le s)$ is true here since $y\ge M_t\ge S_t$; otherwise it could be false as the first hitting time $\tau_t(y)$ after time $t$ does not encode the information of the historical maximum. In several special cases, the first hitting time distribution is known explicitly, thus giving explicit formulae for the optimal trading strategy in feedback form. We shall do so for the case when $S$ is either an arithmetic or a geometric Brownian motion.

\subsection{Arithmetic Brownian motion}
Suppose that $S_t:=S_0(1+ \sigma B_t)$ where $B$ is a standard one-dimensional Brownian motion, and $S_0, \sigma>0$ are constants, so that $L_s(t)$ is the price of a lookback call in the Bachelier model. A straightforward calculation shows that
\[
 \ts \de_s L_s(t) = \frac{S_0\sigma}{\sqrt{s-t}}p\bigl(\frac{c-P_t}{S_0\sigma\sqrt{s-t}}\bigr)\de s,\qquad\text{where } p(x):=\frac{1}{\sqrt{2\pi}}e^{-\frac{1}{2}x^2}.
\]
With this, the optimal selling rate is
\begin{align*}
u^\ast_t &\ts= -\frac{1}{2\lambda G(T-t)}\left(-2\lambda G'(T-t) X_t - \int_t^T \frac{S_0\sigma}{\sqrt{s-t}} G(T-s) p\bigl(\frac{c-P_t}{S_0\sigma\sqrt{s-t}}\bigr) \de s\right) \\
&= \bigl[2\lambda\beta\cosh(\beta (T-t))+2\Gamma\sinh(\beta (T-t))\bigr]^{-1} \\
&\hspace{1.5cm} \times \Bigl[2\left[\Gamma\beta\cosh(\beta (T-t))+\gamma\sinh(\beta (T-t))\right] X_t\\
&\ts\hspace{2.5cm} + \int_t^T \frac{S_0\sigma}{\sqrt{2\pi(s-t)}}\bigl[\beta\cosh(\beta (T-s))+\lambda^{-1}\Gamma\sinh(\beta (T-s))\bigr] \exp{\left(-\frac{1}{2}\frac{(c-P_t)^2}{S_0^2\sigma^2(s-t)}\right)} \de s \Bigr],
\end{align*}
where we recall that $\beta=\sqrt{\lambda^{-1}\gamma}$.

{\color{red}
A straightforward calculation shows that
\begin{align*}
\ts d_s L_s(t) = S_0\sigma^2 p\left(\sigma^2(s-t),\frac{c-P_t}{S_0}\right) ds, \qquad \textrm{ where } p(\theta,x):=(2\pi\theta)^{-1/2}\exp\left(-\frac{x^2}{2\theta}\right).
\end{align*}
With this, the optimal selling rate is
\begin{align*}
u^\ast_t &\ts= \frac{G'(T-t)}{G(T-t)}X_t + \frac{S_0\sigma^2}{2\lambda}\int_t^T \frac{G(T-s)}{G(T-t)}p\left(\sigma^2(s-t),\frac{c-P_t}{S_0}\right) ds \\
&\ts = u^M_t + \frac{S_0\sigma^2}{2\lambda}\int_0^\tau \frac{G(\tau-s)}{G(\tau)}p\left(\sigma^2 s, \frac{c-P_t}{S_0}\right) ds,
\end{align*}
where $\tau:=T-t$ is the time remaining and $u^M_t:=\frac{G'(\tau)}{G(\tau)}X_t$ is the optimal liquidating rate as if the price was a martingale (as in Algrem-Chriss model). Note that $u^M$ is independent of the asset price.
}


%\subsection{Arithmetic Brownian motion}
%Suppose that $S:=B$ where $B$ is a standard one-dimensional Brownian motion. Then by the reflection principle, $\forall s\in[t,T], y\ge M_t, \P_t(\tau_t(y)\le s) = \P^{t,B_t}(\tau_t(y)\le s) = 2\P^{t,B_t}(B_s>y)$.
%
%Therefore, $L_s(t)=\int_{c\vee M_t}2\P^{t,B_t}(B_s > y) dy=2\E^{t,B_t}(B_s-c\vee M_t)_+ = 2u(s-t,B_t; c\vee M_t)$, where $u(\theta,x;K):=\E^{0,x}(B_\theta-K)_+$ is the price of a vanilla European call under the Bachelier model. We take advantage of the partial differential equation (PDE) characterization of the option price:
%\begin{align*}
%u(\theta,x;K) = \left[(\cdot-K)_+ \ast p(\theta,\cdot)\right](x) \iff
%\begin{cases}
%\partial_\theta u - \frac{1}{2}\partial^2_x u =0, & \theta>0, x,K\in\mathbb{R} \\
%u(0,x;K)=(x-K)_+, & x,K\in\mathbb{R} \\
%\abs{u(\theta,x;K)} \le C\sqrt{\theta}(1+\abs{x}+\abs{K}), & \theta\in\mathbb{R}_+, x,K\in\mathbb{R}
%\end{cases},
%\end{align*}
%where $p(\theta,x):=(2\pi\theta)^{-1/2}\exp\left(-\frac{x^2}{2\theta}\right)$ is the transitional probability density kernel of the standard Brownian motion (or the fundamental solution of the heat equation). Thus,
%\begin{align*}
%& d_s L_s(t) = 2\partial_\theta u(s-t,B_t;c\vee M_t) ds = \partial^2_x u(s-t,B_t;c\vee M_t) ds = \left\{\left[\partial^2_x (\cdot-c\vee M_t)_+\right]\ast p(s-t,\cdot)\right\}(B_t) ds \\
%=& \left\{\delta(\cdot-c\vee M_t) \ast p(s-t,\cdot)\right\}(B_t) ds = p(s-t,B_t-c\vee M_t)ds = p(s-t,B_t-(c\vee M_t-c)-c) ds \\
%=& p(s-t, S_t-(M_t-c)_+ -c) ds = p(s-t,P_t-c)ds = p(s-t,c-P_t)ds,
%\end{align*}
%and the optimal selling rate is
%\begin{align*}
%r^\ast_t &= -\frac{1}{2\lambda G(T-t)}\left(-2\lambda G'(T-t) X_t - \int_t^T G(T-s)p(s-t,c-P_t) ds\right) \\
%&= -\left[2\lambda\beta\cosh(\beta (T-t))+2\Gamma\sinh(\beta (T-t))\right]^{-1} \\
%& \cdot \left(- 2\left[\Gamma\beta\cosh(\beta (T-t))+\gamma\sinh(\beta (T-t))\right] X_t - \int_t^T \left[\beta\cosh(\beta (T-s))+\lambda^{-1}\Gamma\sinh(\beta (T-s))\right] \frac{\exp{\left(-\frac{(c-P_t)^2}{2(s-t)}\right)}}{\sqrt{2\pi(s-t)}} ds \right),
%\end{align*}
%where $\beta:=\sqrt{\lambda^{-1}\gamma}$.

\subsection{Geometric Brownian motion}
Let now $S$ be a geometric Brownian motion, i.e.\ $S_t:=S_0\exp\left(\sigma B_t-\frac{1}{2}\sigma^2 t\right)$ for a standard Brownian motion $B$, where $S_0, \sigma >0$ are constants. With this, the computation of the optimal selling rate boils down to the computation of the Theta of a lookback call in the Black-Scholes model. Explicitly, we find that
\[
\ts \de_s L_s(t) = S_t\left[ \frac{\sigma}{\sqrt{s-t}}p\left(\frac{1}{\sigma\sqrt{s-t}}\log\bigl(\frac{c-P_t}{S_t}+1\bigr)-\frac{\sigma\sqrt{s-t}}{2} \right)+\frac{1}{2}\int_{\log\left(\frac{c-P_t}{S_t}+1\right)}^\infty \frac{\sigma}{\sqrt{s-t}}p\left(\frac{1}{\sigma\sqrt{s-t}}z-\frac{\sigma\sqrt{s-t}}{2}\right)\de z \right] \de s.
\]
{\color{blue}Using the change of variables $\tilde z := \frac{1}{\sigma\sqrt{s-t}}z-\frac{\sigma\sqrt{s-t}}{2}$ and the symmetry of $p$, this can equivalently be written as
\begin{align*}
 \ts \de_s L_s(t) &\ts= S_t\left[ \frac{\sigma}{\sqrt{s-t}}p\left(\frac{1}{\sigma\sqrt{s-t}}\log\bigl(\frac{c-P_t}{S_t}+1\bigr)-\frac{\sigma\sqrt{s-t}}{2} \right)+\frac{1}{2}\int_{\log\left(\frac{c-P_t}{S_t}+1\right)}^\infty \frac{\sigma}{\sqrt{s-t}}p\left(\frac{1}{\sigma\sqrt{s-t}}z-\frac{\sigma\sqrt{s-t}}{2}\right)\de z \right] \de s\\
  &\ts = S_t\left[ \frac{\sigma}{\sqrt{s-t}}p\left(\frac{\sigma\sqrt{s-t}}{2}-\frac{1}{\sigma\sqrt{s-t}}\log\bigl(\frac{c-P_t}{S_t}+1\bigr) \right)+\frac{1}{2}\sigma^2 \int_{\frac{1}{\sigma\sqrt{s-t}}\log\left(\frac{c-P_t}{S_t}+1\right) - \frac{\sigma\sqrt{s-t}}{2}}^\infty p(\tilde z)\de\tilde z \right] \de s.
\end{align*}
Now denote by $\mathcal N$ the standard normal cumulative distribution function and set
\[
 \ts f(P_t,S_t,s,t) := \frac{\sigma\sqrt{s-t}}{2} - \frac{1}{\sigma\sqrt{s-t}}\log\left(\frac{c-P_t}{S_t}+1\right).
\]
Then the equation for $\de_s L_s(t)$ simplifies further to
\begin{align*}
 \ts \de_s L_s(t) &\ts= S_t\left[ \frac{\sigma}{\sqrt{s-t}}p\bigl(f(P_t,S_t,s,t)\bigr)+\frac{1}{2}\sigma^2 \bigl(1 - \mathcal N\bigl(-f(P_t,S_t,s,t)\bigr)\bigr)\right] \de s\\
  &\ts= S_t\left[ \frac{\sigma}{\sqrt{s-t}}p\bigl(f(P_t,S_t,s,t)\bigr)+\frac{1}{2}\sigma^2 \mathcal N\bigl(f(P_t,S_t,s,t)\bigr)\right] \de s.
\end{align*}
With this, it follows that the optimal selling rate $r^*$ is given by
\[
 \ts u_t^* = \frac{G'(T-t)}{G(T-t)} X_t + \frac{S_t}{2\lambda}\int_t^T  \frac{G(T-s)}{G(T-t)} \bigl[ \frac{\sigma}{\sqrt{s-t}}p\bigl(f(P_t,S_t,s,t)\bigr)+\frac{1}{2}\sigma^2 \mathcal N\bigl(f(P_t,S_t,s,t)\bigr)\bigr] \de s.
\]
In particular, at time $t=0$ and upon assuming that $P:=P_0=S_0$, this yields
\begin{align*}
 \ts u_0^* &\ts= \frac{G'(T)}{G(T)} x + \frac{P}{2\lambda}\int_t^T  \frac{G(T-s)}{G(T)} \bigl[ \frac{1}{\sqrt{s}}p\bigl(f(P,P,s,0)\bigr)+\frac{1}{2}\mathcal N\bigl(f(P,P,s,0)\bigr)\bigr] \de s\\
           &\ts= \bigl[2\lambda\beta\cosh(\beta T)+2\Gamma\sinh(\beta T)\bigr]^{-1}\\
					 &\ts\hspace{1.5cm}\times\Bigl[ 2\bigl[\Gamma\beta\cosh(\beta T)+\gamma\sinh(\beta T)\bigr] x\\
					 &\ts\hspace{2.5cm}+ P\int_0^T  \bigl[\beta\cosh(\beta (T-s))+\lambda^{-1}\Gamma\sinh(\beta (T-s))\bigr] \bigl[ \frac{\sigma}{\sqrt{s}}p\bigl(\frac{\sigma\sqrt{s}}{2} - \frac{1}{\sigma\sqrt{s}}\log\left(\frac{c}{P}\right)\bigr)\\
					 &\ts\hspace{9.5cm}+\frac{1}{2}\sigma^2 \mathcal N\bigl(\frac{\sigma\sqrt{s}}{2} - \frac{1}{\sigma\sqrt{s}}\log\left(\frac{c}{P}\right)\bigr)\bigr] \de s\Bigr].
\end{align*}
}

{\color{red}
For the sake of numerical implementation (e.g. in Matlab), we express our results as follows.
\begin{multline*}
\ts \de_s L_s(t) = S_t\sigma^2 \Bigl[p\left(\log\left(\frac{c-P_t}{S_t}+1 \right); \frac{1}{2}\sigma^2(s-t), \sigma^2(s-t)\right)\\
   \ts + \frac{1}{2}\left(1-\mathcal{N}\left(\log\left(\frac{c-P_t}{S_t}+1 \right); \frac{1}{2}\sigma^2(s-t), \sigma^2(s-t)\right)\right)\Bigr] \de s,
\end{multline*}
where $p(\cdot;\mu,\Sigma)$ and $\mathcal{N}(\cdot;\mu, \Sigma)$ are respectively the probability density function and the cumulative distribution function of a normal random variable with mean $\mu\in\mathbb{R}$ and variance $\Sigma>0$.

With $\tau$ and $u^M$ as in the case of arithmetic Brownian motion, and further introducing the ``log-moneyness'' variable $l:=\log\left(\frac{c-P_t}{S_t}+1\right)$, we have
\begin{align*}
\ts u^\ast_t = u^M_t + \frac{S_t\sigma^2}{2\lambda}\int_0^\tau \frac{G(\tau-s)}{G(\tau)}\left[p(l;\frac{1}{2}\sigma^2 s, \sigma^2 s)+\frac{1}{2}(1-\mathcal{N}(l;\frac{1}{2}\sigma^2 s, \sigma^2 s))\right] \de s.
\end{align*}
}


%\subsection{Geometric Brownian motion}
%Now suppose that $S$ is a standard geometric Brownian motion, i.e. $S_t:=\mathcal{E}(B)_t = \exp\left(B_t-\frac{1}{2}t\right)$ in which $B$ is a standard Brownian motion. We would like to compute $\P_t(\tau_t(y)\le s) = \P^{0,S_t}(\tau_0(y)\le s-t)$ for $y>M_t$.
%
%For simplicity, we will compute $\P^{0,x}(\tau_0(y)\le \theta)$ for $y>x$. Define $\tilde{B}_t:=B_t-\frac{1}{2}t$ and $Z_t:=\exp\left(-\frac{1}{2}\tilde{B}_t-\frac{1}{8}t\right)$. Set $\tilde\tau(z):=\inf\{t\in[0,\infty] : \tilde{B}_t = z\}$. Let $\tilde\P$ be such that $\frac{d\P}{d\tilde\P}=Z_\theta$, then by Girsanov's theorem, $\{\tilde{B}_t\}_{t\in[0,\theta]}$ is a Brownian motion under $\tilde\P$. Thus, with knowledge of the first hitting time of a standard Brownian motion,
%\begin{align*}
%& \P^{0,x}(\tau_0(y)\le \theta) = \P(\tilde\tau(\log(y/x)) \le \theta) = \tilde\E\left[Z_\theta \I_{\{\tilde\tau(\log(y/x))\le \theta\}}\right] = \tilde\E\left[Z_{\tilde\tau(\log(y/x))} \I_{\{\tilde\tau(\log(y/x))\le \theta\}}\right] \\
%=& \tilde\E\left[\exp\left(-\frac{1}{2}\log(y/x)-\frac{1}{8}\tilde\tau(\log(y/x))\right) \I_{\{\tilde\tau(\log(y/x))\le \theta\}}\right] = \int_0^\theta (y/x)^{-1/2}e^{-t/8} d_t\tilde\P(\tilde\tau(\log(y/x)) \le t) \\
%=& \int_0^\theta (y/x)^{-1/2}e^{-t/8} d_t\left(2\tilde\P(\tilde{B}_t>\log(y/x))\right) = \int_0^\theta (y/x)^{-1/2}e^{-t/8} d_t\left(2\int_{\log(y/x)}^\infty p(t,z) dz\right) \\
%=& \int_0^\theta (y/x)^{-1/2}e^{-t/8} \left(\int_{\log(y/x)}^\infty 2\partial_t p(t,z) dz\right) dt = \int_0^\theta (y/x)^{-1/2}e^{-t/8} \left(\int_{\log(y/x)}^\infty \partial^2_z p(t,z) dz\right) dt \\
%=& -\int_0^\theta (y/x)^{-1/2}e^{-t/8} \partial_z p(t,\log(y/x)) dt, \quad \textrm{ where } p(t,z):=(2\pi t)^{-1/2}\exp\left(-\frac{z^2}{2t}\right).
%\end{align*}
%Therefore, let $z:=\log(y/x)$ we have
%\begin{align*}
%& \int_K^\infty \P^{0,x}(\tau_0(y)\le \theta) dy = -\int_{\log(K/x)}^\infty \int_0^\theta e^{-z/2} e^{-t/8} \partial_z p(t,z) dt d(xe^{z}) = -x \int_0^\theta e^{-t/8} \int_{\log(K/x)}^\infty e^{z/2} \partial_z p(t,z)) dz dt \\
%=& x \int_0^\theta e^{-t/8} \left[(K/x)^{1/2} p(t,\log(K/x)) + \int_{\log(K/x)}^\infty \frac{1}{2}e^{z/2}p(t,z) dz\right] dt \\
%=& x \int_0^\theta \left[p(t,\log(K/x)-t/2) + \frac{1}{2}\int_{\log(K/x)}^\infty p(t,z-t/2) dz\right] dt.
%\end{align*}
%With $x=S_t, K=c\vee M_t=(c\vee M_t -c - S_t) + c + S_t =c-P_t+S_t$, and $\theta=s-t$, we obtain
%\begin{align*}
%& L_s(t)=S_t \int_0^{s-t} \left[ p\left(\tau,\log\left(\frac{c-P_t}{S_t}+1\right)-\tau/2 \right)+\frac{1}{2}\int_{\log\left(\frac{c-P_t}{S_t}+1\right)}^\infty p(\tau,z-\tau/2)dz \right] d\tau \\
%\Rightarrow & d_s L_s(t) = S_t\left[ p\left(s-t,\log\left(\frac{c-P_t}{S_t}+1\right)-\frac{s-t}{2} \right)+\frac{1}{2}\int_{\log\left(\frac{c-P_t}{S_t}+1\right)}^\infty p\left(s-t,z-\frac{s-t}{2}\right)dz \right] ds.
%\end{align*}
%In short, we also have an explicit formula for the optimal selling rate if $S$ is a geometric Brownian motion. 

%\subsection{It\^o Diffusion}
%If $S$ is a diffusion given by the stochastic differential equation (SDE) $dS_t = \sigma(t,S_t) dB_t$, where $\sigma$ is uniformly Lipschitz in space and $B$ is a standard Brownian motion (note that $S$ is a martingale), then the cumulative distribution function (CDF) of the first hitting time $\P^{t,x}(\tau_t(y)\le s)=:u(t,x;s,y)$ can be characterized as the bounded solution to the following linear heat equation:
%\begin{align*}
%\begin{cases}
%(\partial_t + \mathcal{L}^S)u = \partial_t u + \frac{1}{2}\sigma^2\partial^2_x u = 0, & t<s, x<y \\
%u(t,y;s,y)=1, & t<s \\
%u(s,x;s,y)=\I_{\{x=y\}}, & x<y
%\end{cases}.
%\end{align*}
%Then we can write $L_s(t) = \int_{c\vee M_t} u(t,B_t;s,y) dy$.
%
%Now that one has to solve a PDE for diffusive $S$, we can also directly obtain a PDE for $L_s(t)$. Set $u(t,x,y;s,K):=\E\left[(M_s-K)_+ \vert S_t=x, M_t=y\right]$, then $u$ solves the following linear heat equation:
%\begin{align*}
%\begin{cases}
%(\partial_t u + \mathcal{L}^S)u = \partial_t u + \frac{1}{2}\sigma^2\partial^2_x u = 0, & t<s, x<y \\
%\partial_y u(t,y,y;s,K) = 0, & t<s \\
%u(s,x,y;s,K) = (y-K)_+, & x<y
%\end{cases}
%\end{align*}
%Then we can write $L_s(t) = u(t,B_t,M_t;s,c\vee M_t)$.

%\subsection{Price floor and model assumptions}
%Our proposition also applies to the case of liquidating a positive position of an asset in the presence of a price floor, and in particular, there is a unique optimal strategy absolutely continuous with respect to time. Formally, let $P=S+(f-M)_+$ be the asset price in which $S\in\mathcal{H}^1$ is a martingale and $M_t:=\inf_{u\in[0,t]}S_u$ is the running minimum. This is the dual case of an asset of price $-P$ and a price cap at $-f$. Thus, $P_t\ge f$ with equality iff $S_t=M_t\le f$, and $E_t P_s = P_t + \E_t[(f\wedge M_t-M_s)_+]$ from the beginning of this section. Therefore, one can solve the optimal liquidation problem with a price floor by pricing fixed-strike lookback puts. In the case of $S=B$, a standard Brownian motion, the optimal selling rate is
%\begin{align*}
%r^\ast_t &= -\frac{1}{2\lambda G(T-t)}\left(-2\lambda G'(T-t) X_t + \int_t^T G(T-s)p(s-t,P_t-f) ds\right) \\
%&= -\left[2\lambda\beta\cosh(\beta (T-t))+2\Gamma\sinh(\beta (T-t))\right]^{-1} \\
%& \cdot \left(- 2\left[\Gamma\beta\cosh(\beta (T-t))+\gamma\sinh(\beta (T-t))\right] X_t + \int_t^T \left[\beta\cosh(\beta (T-s))+\lambda^{-1}\Gamma\sinh(\beta (T-s))\right] \frac{\exp{\left(-\frac{(P_t-f)^2}{2(s-t)}\right)}}{\sqrt{2\pi(s-t)}} ds \right),
%\end{align*}
%Note that the sign of the integral term is flipped.
%
%However, it makes little economic sense to interpret $\lambda r_t$ as the temporary price impact of selling $r_t dt$ units of the asset at the price floor $f$, because by definition there should be no downward price impact at the floor. In this case, one could argue that it is a trading cost from commissions or exchange fees instead of temporary price impact, but in reality, these costs are usually proportional to the total shares traded or even decreasing in trading intensities. Therefore, we think that our model is not suitable for asset liquidation in the presence of a price floor.

% references
\bibliographystyle{plainnat}
\phantomsection
\addcontentsline{toc}{section}{\refname}
\bibliography{References}

\end{document}